\section{The ATLAS detector}
\label{sec:ATLAS}

The ATLAS experiment~\cite{Collaboration_2008} is a multi-purpose particle detector with a forward-backward symmetric cylindrical geometry and nearly $4\pi$ coverage in sold angle.
The collision point is surrounded by inner tracking devices followed by a superconducting solenoid providing a 2T magnetic field, a calorimeter system, and a muon spectrometer.

The inner tracker provides precision tracking of charged particles for pseudorapidities $|\eta|<2.5$.
It consists os pixel and silicon-microstrip detectors inside of a transition radiation tracker.
One significant upgrade for the new 13~TeV running period is the presence of the Insertable B-Layer~\cite{Capeans:1291633}, an additional pixel layer that provides high-resolution hits at small radius to improve tracking performance.

In the pseudorapidity region $|\eta|<3.2$, high-granularity lead/liquid-argon (LAr) electromagnetic (EM) sampling calorimeters are used.
An iron/scintillator tile calorimeter measures hadron energies for $|\eta|<1.7$.
The endcap and forward regions, spanning $1.5 < |\eta| < 4.9$, are instrumented with LAr calorimeters for both EM and hadronic measurements.

The muon spectrometer consists of three large superconducting toroids with 8 coils each, a system of trigger chambers, and precision tracking chambers,  which provide triggering and tracking capabilities in the ranges $|\eta|<2.4$ and $|\eta|<2.7$, respectively.

The two level trigger system is used to select events.
The first-level trigger is implemented in hardware and uses a subset of the detector information.
This is followed by the software-based High-Level Trigger (HLT) system, which can run offline reconstruction and calibration software, reducing the event rate to less than 1 kHz.